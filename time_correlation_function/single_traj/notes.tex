\documentclass{article}
%==============================================================================%
%                                 Packages                                     %
%==============================================================================%
% Packages
\usepackage[utf8]{inputenc}
\usepackage{graphicx}
\usepackage{float}
\usepackage{amsmath}
\usepackage{amssymb}
\usepackage{braket}
\usepackage{subcaption}
\usepackage[margin=0.7in]{geometry}
\usepackage[version=4]{mhchem}
%==============================================================================%
%                           User-Defined Commands                              %
%==============================================================================%
% User-Defined Commands
\newcommand{\be}{\begin{equation}}
\newcommand{\ee}{\end{equation}}
\newcommand{\benum}{\begin{enumerate}}
\newcommand{\eenum}{\end{enumerate}}
\newcommand{\pd}{\partial}
\newcommand{\dg}{\dagger}
\newcommand{\ha}{\hat{A}}
\newcommand{\hb}{\hat{B}}
\newcommand{\hc}{\hat{C}}
%==============================================================================%
%                             Title Information                                %
%==============================================================================%
\title{Position Autocorrelation Function for a Harmonic Oscillator}
\date{\today}
\author{Alan Robledo}
\begin{document}
\maketitle
As always we define the Hamiltonian for the 1D system.
\be
  H = \frac{p^2}{2m} + \frac{1}{2} m \omega^2 x^2
\ee
In the canonical ensemble, $C_{xx}(t)$ for a 1D Harmonic Oscillator has a simple form,
\be
  C_{xx}(t) = \frac{kT}{m \omega^2} \cos(\omega t)
\ee
where $k$ is Boltzmann's constant and $T$ is temperature. Computing this value numerically with a Molecular Dynamics trajectory poses an issue because the correlation function requires that the dynamics be realistic --- something that cannot be generated for a system coupled to a thermal bath. We can overcome this issue by defining the correlation function in the microcanonical ensemble.

The first thing we need to do is compute the partition function for a Harmonic Oscillator in the microcanonical ensemble, i.e. we need to compute,
\be
  \Omega = \frac{E_o}{h} \int_{-\infty}^{\infty} dp \int_{-\infty}^{\infty} dx \; \delta(H(x,p) - E) = \frac{E_o}{h} \int_{-\infty}^{\infty} dp \int_{-\infty}^{\infty} dx \; \delta\left( \frac{p^2}{2m} + \frac{1}{2} m \omega^2 x^2 - E \right)
\ee
which is non-trivial but doable. We begin by defining some new coordinates,
\be \label{eq:bar_coords}
  \begin{split}
    \bar{p} &= \frac{p}{\sqrt{2m}} \quad ; \quad \sqrt{2m} d\bar{p} = dp \\
    \bar{x} &= \sqrt{ \frac{m \omega^2}{2} } x \quad ; \quad \sqrt{ \frac{2}{m \omega^2} } d\bar{x} = dx
  \end{split}
\ee
and substituting them into the partition function.
\be
  \Omega = \frac{E_o}{h} \sqrt{2m} \sqrt{ \frac{2}{m \omega^2} } \int_{-\infty}^{\infty} d \bar{p} \int_{-\infty}^{\infty} d\bar{x} \; \delta\left( \bar{p}^2 + \bar{x}^2 - E \right) = \frac{2 E_o}{h \omega} \int_{-\infty}^{\infty} d \bar{p} \int_{-\infty}^{\infty} d\bar{x} \; \delta\left( \bar{p}^2 + \bar{x}^2 - E \right)
\ee
The delta function requires that we consider points where $p^2 + x^2 = E$ which resembles the equation for a circle so we can consider a conversion to polar coordinates.
\be \label{eq:polar_coords}
  \begin{split}
    \bar{p} &= \sqrt{r \omega} \cos(\theta) \\
    \bar{x} &= \sqrt{r \omega} \sin(\theta)
  \end{split}
\ee
We define the coordinates this way so that the jacobian is simply a factor of $\omega$.
\be
  |J| =
  \begin{vmatrix}
    \frac{\partial \bar{p}}{\partial r} & \frac{\partial \bar{p}}{\partial \theta}\\[6pt]
    \frac{\partial \bar{x}}{\partial r} & \frac{\partial \bar{x}}{\partial \theta}
  \end{vmatrix}
  =
  \begin{vmatrix}
    \sqrt{\omega} \cos(\theta) / 2\sqrt{r} & - \sqrt{r \omega} \sin(\theta) \\
    \sqrt{\omega} \sin(\theta) / 2\sqrt{r} & \sqrt{r \omega} \cos(\theta) \\
  \end{vmatrix}
  = \frac{\omega}{2} \cos^2(\theta) + \frac{\omega}{2} \sin^2(\theta) = \frac{\omega}{2}
\ee
The integrals become redefined as,
\be
  \int_{-\infty}^{\infty} d \bar{p} \int_{-\infty}^{\infty} d\bar{x} = \int_{0}^{2 \pi} d \theta \int_{0}^{\infty} d r |J|  = \frac{\omega}{2} \int_{0}^{2 \pi} d \theta \int_{0}^{\infty} d r
\ee
Equation (5) becomes,
\be
  \Omega = \frac{E_o}{h} \int_{0}^{2 \pi} d \theta \int_{0}^{\infty} d r \; \delta\left( r \omega - E \right)
\ee
The integral over $\theta$ is trivial,
\be
  \Omega = \frac{2 \pi E_o}{h} \int_{0}^{\infty} d r \; \delta\left( r \omega - E \right)
\ee
and we can set $r' = r \omega$,
\be
  \Omega = \frac{2 \pi E_o}{h \omega} \int_{0}^{\infty} d r' \; \delta\left( r' - E \right)
\ee
to get an integral of a dirac delta function over all space which is equal to 1. So we are left with,
\be
  \Omega = \frac{2 \pi E_o}{h \omega} = \frac{E_o}{\hbar \omega} .
\ee

Now we can begin deriving a form for the correlation function.
We start with the definition of $C_{xx}(t)$ in the microcanonical ensemble (we take $t=0$ to be our initial time) as a phase space average,
\be
  C_{xx}(t) = \braket{x(0) x(t)} = \frac{E_o}{h \Omega} \int_{-\infty}^{\infty} dp \int_{-\infty}^{\infty} dx \; x(0) \; x(t) \delta(H(x,p) - E)
\ee
where $E_o/h$ is a constant from the partition function $\Omega$.
Integrating Hamilton's equations for a Harmonic Oscillator yields the equation of motion
\be
  x(t) = x(0) \cos(\omega t) + \frac{p(0)}{m \omega} \sin(\omega t)
\ee
where $p(0)$ is the initial momentum. We can drop the 0 in the initial x and p since we need to consider each point in phase space as an initial condition in order to compute the integral. Plugging $x(t)$ and equation (12) into our defintion leaves us with,
\be
  C_{xx}(t) = \frac{\omega}{2 \pi} \int_{-\infty}^{\infty} dp \int_{-\infty}^{\infty} dx \; x \left[ x \cos(\omega t) + \frac{p}{m \omega} \sin(\omega t) \right] \delta \left( \frac{p^2}{2m} + \frac{1}{2} m \omega^2 x^2 - E \right)
\ee
We will now make the same transformation as in equation (\ref{eq:bar_coords}) to make $(x,p) \rightarrow (\bar{x}, \bar{p})$.
\be
\begin{split}
  C_{xx}(t) &= \frac{\omega}{2 \pi} \sqrt{2m} \sqrt{ \frac{2}{m \omega^2} } \int_{-\infty}^{\infty} d \bar{p} \int_{-\infty}^{\infty} d \bar{x} \; \sqrt{\frac{2}{m \omega^2}} \bar{x} \left[ \sqrt{\frac{2}{m \omega^2}} \bar{x} \cos(\omega t) + \frac{\sqrt{2m}}{m \omega} \bar{p} \sin(\omega t) \right] \delta \left( \bar{p}^2 + \bar{x}^2 - E \right) \\
  &= \frac{\omega}{2 \pi} \frac{2}{\omega} \int_{-\infty}^{\infty} d \bar{p} \int_{-\infty}^{\infty} d \bar{x} \; \sqrt{\frac{2}{m \omega^2}} \bar{x} \left[ \sqrt{\frac{2}{m \omega^2}} \bar{x} \cos(\omega t) + \sqrt{\frac{2}{m \omega^2}} \bar{p} \sin(\omega t) \right] \delta \left( \bar{p}^2 + \bar{x}^2 - E \right) \\
  &= \frac{1}{\pi} \int_{-\infty}^{\infty} d \bar{p} \int_{-\infty}^{\infty} d \bar{x} \; \frac{2}{m \omega^2} \bar{x} \left[ \bar{x} \cos(\omega t) + \bar{p} \sin(\omega t) \right] \delta \left( \bar{p}^2 + \bar{x}^2 - E \right) \\
  &= \frac{2}{ \pi m \omega^2} \int_{-\infty}^{\infty} d \bar{p} \int_{-\infty}^{\infty} d \bar{x} \; \bar{x} \left[ \bar{x} \cos(\omega t) + \bar{p} \sin(\omega t) \right] \delta \left( \bar{p}^2 + \bar{x}^2 - E \right) \\
\end{split}
\ee
And then make the second transformation to polar coordinates as in equation (\ref{eq:polar_coords}).
\be
  \begin{split}
    C_{xx}(t) &= \frac{2}{ \pi m \omega^2} \frac{\omega}{2} \int_{0}^{2 \pi} d \theta \int_{0}^{\infty} d r \; \sqrt{r \omega} \sin(\theta) \left[ \sqrt{r \omega} \sin(\theta) \cos(\omega t) + \sqrt{r \omega} \cos(\theta) \sin(\omega t) \right] \delta \left( r \omega - E \right) \\
    &= \frac{1}{ \pi m \omega} \int_{0}^{2 \pi} d \theta \int_{0}^{\infty} d r \; r \omega \sin(\theta) \left[ \sin(\theta) \cos(\omega t) + \cos(\theta) \sin(\omega t) \right] \delta \left( r \omega - E \right) \\
  \end{split}
\ee
The following trig identity, $\sin(x)\cos(y) + \sin(y)\cos(x) = \sin(x+y)$, yields,
\be
\begin{split}
  C_{xx}(t) &= \frac{1}{ \pi m} \int_{0}^{2 \pi} d \theta \int_{0}^{\infty} d r \; r \sin(\theta) \left[ \sin(\theta + \omega t) \right] \delta \left( r \omega - E \right) \\
  &= \frac{1}{ \pi m} \int_{0}^{\infty} d r \; r \; \delta \left( r \omega - E \right) \int_{0}^{2 \pi} d \theta \sin(\theta) \sin(\theta + \omega t)
\end{split}
\ee
The theta integral is trivial.
\be
\begin{split}
  \int_{0}^{2 \pi} d \theta \sin(\theta) \sin(\theta + \omega t) &= \int_{0}^{2 \pi} d \theta \sin(\theta) \left[ \sin(\theta) \cos(\omega t) + \cos(\theta) \sin(\omega t) \right] \\
  &= \cos(\omega t) \int_{0}^{2 \pi} d \theta \sin^2 (\theta) + \sin(\omega t) \int_{0}^{2 \pi} d \theta \sin(\theta) \cos(\theta) \\
  &= \cos(\omega t) \int_{0}^{2 \pi} d \theta \frac{1}{2} \left[ 1 - \cos(2 \theta)\right] + \sin(\omega t) \int_{0}^{2 \pi} d \theta \sin(\theta) \cos(\theta) \\
  &= \frac{\cos(\omega t)}{2} \int_{0}^{2 \pi} d \theta \left[ 1 - \cos(2 \theta)\right] + \sin(\omega t) \int_{0}^{0} du \; u \\
  &= \frac{\cos(\omega t)}{2} \left( \theta - \frac{\sin(2 \theta)}{2} \right) \bigg|^{2 \pi}_0 + 0\\
  &= \frac{\cos(\omega t)}{2} (2\pi) \\
  &= \pi \cos(\omega t)
\end{split}
\ee
So we are left with,
\be
  C_{xx}(t) = \frac{1}{m} \int_{0}^{\infty} d r \; r \cos(\omega t) \delta \left( r \omega - E \right)
\ee
If we set $r' = r \omega$ we are left with another delta function that is simple to evaluate with the following identity, $\int_{-\infty}^{\infty} f(x) \delta(x - a) = f(a)$, (remembering that $r \in [0, \infty)$)
\be
\begin{split}
  C_{xx}(t) &= \frac{1}{m} \int_{0}^{\infty} \frac{d r'}{\omega} \; \frac{r'}{\omega} \cos(\omega t) \delta \left( r' - E \right) \\
  &= \frac{1}{m \omega^2} \cos(\omega t) \int_{0}^{\infty} dr' \; r' \delta \left( r' - E \right) \\
  &= \frac{1}{m \omega^2} \cos(\omega t) E
\end{split}
\ee
And we finally derive the position autocorrelation function in the microcanonical ensemble,
\be
  C_{xx}(t) = \frac{E}{m \omega^2} \cos(\omega t)
\ee
where $E$ is the constant total energy of the system.
\end{document}
