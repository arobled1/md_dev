\documentclass{article}
%==============================================================================%
%                                 Packages                                     %
%==============================================================================%
% Packages
\usepackage[utf8]{inputenc}
\usepackage{graphicx}
\usepackage{float}
\usepackage{amsmath}
\usepackage{amssymb}
\usepackage{braket}
\usepackage{subcaption}
\usepackage[margin=0.7in]{geometry}
\usepackage[version=4]{mhchem}
%==============================================================================%
%                           User-Defined Commands                              %
%==============================================================================%
% User-Defined Commands
\newcommand{\be}{\begin{equation*}}
\newcommand{\ee}{\end{equation*}}
\newcommand{\benum}{\begin{enumerate}}
\newcommand{\eenum}{\end{enumerate}}
\newcommand{\pd}{\partial}
\newcommand{\dg}{\dagger}
\newcommand{\ha}{\hat{A}}
\newcommand{\hb}{\hat{B}}
\newcommand{\hc}{\hat{C}}
%==============================================================================%
%                             Title Information                                %
%==============================================================================%
\title{Position Autocorrelation Function for a Harmonic Oscillator}
\date{\today}
\author{Alan Robledo}
\begin{document}
\maketitle
As always we define the Hamiltonian for the 1D system.
\be
  H = \frac{p^2}{2m} + \frac{1}{2} m \omega^2 x^2
\ee
In the canonical ensemble, $C_{xx}(t)$ for a 1D HO has a simple form,
\be
  C_{xx}(t) = \frac{kT}{m \omega^2} \cos(\omega t)
\ee
where $k$ is Boltzmann's constant and $T$ is temperature. Computing this value numerically with a Molecular Dynamics trajectory poses an issue because the correlation function requires that the dynamics be realistic - something that cannot be generated for a system coupled to a thermal bath. We can overcome this issue by defining the correlation function in the microcanonical ensemble.

The first thing we need to do is compute the partition function for a Harmonic Oscillator in the microcanonical ensemble, i.e. we need to compute,
\be
  \Omega = \frac{E_o}{h} \int_{-\infty}^{\infty} dp \int_{-\infty}^{\infty} dx \; \delta(H(x,p) - E) = \frac{E_o}{h} \int_{-\infty}^{\infty} dp \int_{-\infty}^{\infty} dx \; \delta\left( \frac{p^2}{2m} + \frac{1}{2} m \omega^2 x^2 - E \right)
\ee
which is non-trivial but doable. We begin by defining some new coordinates,
\be
  \begin{split}
    \bar{p} &= \frac{p}{\sqrt{2m}} \quad \quad \sqrt{2m} d\bar{p} = dp \\
    \bar{x} &= \sqrt{ \frac{m \omega^2}{2} } x \quad \quad \sqrt{ \frac{2}{m \omega^2} } d\bar{x} = dx
  \end{split}
\ee
and substitute into the partition function.
\be
  \Omega = \frac{E_o}{h} \sqrt{2m} \sqrt{ \frac{2}{m \omega^2} } \int_{-\infty}^{\infty} d \bar{p} \int_{-\infty}^{\infty} d\bar{x} \; \delta\left( \bar{p}^2 + \bar{x}^2 - E \right) = \frac{2 E_o}{h \omega} \int_{-\infty}^{\infty} d \bar{p} \int_{-\infty}^{\infty} d\bar{x} \; \delta\left( \bar{p}^2 + \bar{x}^2 - E \right)
\ee
The delta function requires that we consider points where $p^2 + x^2 = E$ which resembles the equation for a circle so we can consider a conversion to polar coordinates.
\be
  \begin{split}
    \bar{p} &= \sqrt{r \omega} \cos(\theta) \\
    \bar{x} &= \sqrt{r \omega} \sin(\theta)
  \end{split}
\ee
We define the coordinates this way so that the jacobian equals $\omega$ when redefining the integrals.
\be
  |J| =
  \begin{vmatrix}
    \frac{\partial \bar{p}}{\partial r} & \frac{\partial \bar{p}}{\partial \theta}\\[6pt]
    \frac{\partial \bar{x}}{\partial r} & \frac{\partial \bar{x}}{\partial \theta}
  \end{vmatrix}
  =
  \begin{vmatrix}
    \sqrt{\omega} \cos(\theta) / 2\sqrt{r} & - \sqrt{r \omega} \sin(\theta) \\
    \sqrt{\omega} \sin(\theta) / 2\sqrt{r} & \sqrt{r \omega} \cos(\theta) \\
  \end{vmatrix}
  = 
\ee
\be
  \int_{-\infty}^{\infty} d \bar{p} \int_{-\infty}^{\infty} d\bar{x} = \int_{-0}^{2 \pi} d \theta \int_{0}^{\infty} d r |J|
\ee

% We start with the definition of $C_{xx}(t)$ in the microcanonical ensemble (we take $t=0$ to be our initial time) as a phase space average,
% \be
%   C_{xx}(t) = \braket{x(0) x(t)} = \frac{E_o}{h \Omega} \int_{-\infty}^{\infty} dp \int_{-\infty}^{\infty} dx \; x(0) \; x(t) \delta(H(x,p) - E)
% \ee
% where $E_o/h$ is a constant from the partition function $\Omega$.
% Integrating Hamilton's equations for a Harmonic Oscillator yields the equation of motion
% \be
%   x(t) = x(0) \cos(\omega t) + \frac{p(0)}{m \omega} \sin(\omega t)
% \ee
% where $p(0)$ is the initial momentum. We can drop the 0 in the initial x and p since we need to consider each point in phase space as an initial condition in order to compute the integral. Plugging $x(t)$ into our defintion leaves us with,
% \be
%   C_{xx}(t) = \frac{E_o}{h \Omega} \int_{-\infty}^{\infty} dp \int_{-\infty}^{\infty} dx \; x \left[ x \cos(\omega t) + \frac{p}{m \omega} \sin(\omega t) \right] \delta \left( \frac{p^2}{2m} + \frac{1}{2} m \omega^2 x^2 - E \right)
% \ee

\end{document}
