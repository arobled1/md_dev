\documentclass{article}
%=============================================================================80
%	                          Packages                                     %
%==============================================================================%
% Packages
\usepackage[utf8]{inputenc}
\usepackage{graphicx}
\usepackage{amsmath}
\usepackage{amssymb}
\usepackage{braket}
\usepackage{float}
\usepackage{subcaption}
\usepackage[margin=0.7in]{geometry}
\usepackage[version=4]{mhchem}
%==============================================================================%
%                           User-Defined Commands                              %
%==============================================================================%
% User-Defined Commands
\newcommand{\be}{\begin{equation}}
\newcommand{\ee}{\end{equation}}
\newcommand{\benum}{\begin{enumerate}}
\newcommand{\eenum}{\end{enumerate}}
\newcommand{\pd}{\partial}
\newcommand{\dg}{\dagger}
%==============================================================================%
%                             Title Information                                %
%==============================================================================%
\title{}
\date{11/14/19}
\author{Alan Robledo}
%==============================================================================%
\begin{document}

\maketitle

\section{Derivation of Path Integral for 1D Harmonic Oscillator}
As always, we will start with the case of a single quantum particle moving in one dimension. So the Hamiltonian looks like,
\be
  H = \frac{p^2}{2m} + U(x)
\ee
We consider the amplitude of a path that the particle can take when going from x to x' over some time t as the propagator written in position space.
\be
  A = \braket{x'|e^{-iHt/\hbar}|x}
\ee
This is useful to know because the probability of the particle choosing a particular path would simply be $|A|^2$.
In terms of state vectors, the initial state of our particle can be represented as $\ket{\Psi(0)}$, so the state of our particle can be represented as the inital state vector being acted upon by the propagator
\be
  \ket{\Psi(t)} = e^{-iHt/\hbar} \ket{\Psi(0)}
\ee
And we can project the coordinate basis onto our state vectors to give us a meaningful representation of our state vector
\be
  \begin{split}
    \braket{x'|\Psi(t)} = \Psi(x',t) &= \braket{x'|e^{-iHt/\hbar}|\Psi(0)} \\
    &= \braket{x'|e^{-iHt/\hbar}\int dx|x} \braket{x|\Psi(0)} \\
    &= \int dx \braket{x'|e^{-iHt/\hbar}|x} \braket{x|\Psi(0)} \\
    &= \int dx \braket{x'|e^{-iHt/\hbar}|x} \Psi(x,0)
  \end{split}
\ee
where we introduced the resolution of identity $\int dx \ket{x} \bra{x} = 1$.

So now we have a more mathematical representation of our problem; we want to know how to go from $\Psi(x,0)$ to $\Psi(x',t)$.
The physical representation of our problem is that we have a particle at some point x in space, and we want to be able to detect our particle at some other point x' in space after some time t passes.

By going to imaginary time, which is done by substituting $it/\hbar = \beta$, we can express the coordinate-space quantum propagator as the canonical density matrix,
\be \label{eq:density_mat}
  \rho(x,x') = \braket{x'|e^{-\beta \hat{H}}| x}
\ee
The motive behind doing this is so that we can deal with a damped exponential as opposed to a complex exponential.
Therefore, we can derive the path integral form of the density matrix and simply resubstitute time in place of $\beta$ to obtain the path integral form of the time propagator.

We would like to separate the hamiltonian into its kinetic and potential energy components, $\hat{H} = \hat{K} + \hat{U}$.
However, we cannot split the exponentials because for two operators that do not commute $\text{exp}(\hat{H}) \neq \text{exp}(\hat{K}) \text{exp}(\hat{U})$.
To get around this we employ the trotter decompositon
\be
  e^{- \beta \hat{H}} = e^{- \beta (\hat{K} + \hat{U})} = \lim_{P \to\infty} \Big[ e^{- \beta \hat{U}/2P} e^{-\beta \hat{K}/P} e^{-\beta{U}/2P} \Big]^P .
\ee
Substituting this into equation (\ref{eq:density_mat}) gives us,
\be
  \rho(x,x') = \lim_{P \to\infty} \bra{x'} \Big[ e^{- \beta \hat{U}/2P} e^{-\beta \hat{K}/P} e^{-\beta{U}/2P} \Big]^P \ket{x}
\ee
which can be condensed by defining another operator $\Omega$ as
\be
  \Omega = e^{-\beta \hat{U}/2P} e^{-\beta \hat{K}/P} e^{-\beta \hat{U}/2P} .
\ee
So our density matrix becomes,
\be
  \rho(x,x') = \lim_{P \to\infty} \braket{x'|\Omega^P|x}
\ee
We can insert the resolution of identity in terms of the position basis P-1 times between each of the omegas, remembering that $\Omega^P = \Omega \Omega \cdots \Omega$,
\be \label{eq:density_omega}
  \rho(x,x') = \lim_{P \to\infty} \int \cdots \int dx_P \cdots dx_2 \braket{x'|\Omega|x_P} \braket{x_P|\Omega|x_{P-1}} \bra{x_{P-1}} \cdots \ket{x_3} \braket{x_3|\Omega|x_2} \braket{x_2|\Omega|x}
\ee
The integration over a coordinate $x_i$ can be thought of as integrating over a possible path that your particle can take because we can obtain integrations over all possible paths when P $\rightarrow \infty$.
If we consider an element of the matrix,
\be \label{eq:omega_mat}
  \braket{x_{k+1}|\Omega|x_k} = \braket{x_{k+1}|e^{-\beta \hat{U}/2P} e^{-\beta \hat{K}/P} e^{-\beta \hat{U}/2P}|x_k}
\ee
we can recognize that the set of $\ket{x_k}$ are eigenvectors of $e^{-\beta \hat{U}/2P}$ with eigenvalue $e^{-\beta U(x_k)}/2P$ because $\hat{U} = U(\hat{x})$ is a function of the coordinate operator.
This can be shown easily by writing the exponential as a power series in $\hat{U}$.
\be
  e^{-\beta \hat{U}/2P} = \sum_{n=0}^{\infty} \frac{\beta^n(2P)^{-n} \hat{U}^n}{n!}
\ee
and if we operate on an eigenvector $\ket{x_k}$, we get
\be
  \begin{split}
    e^{-\beta \hat{U}/2P} \ket{x_k} &= \sum_{n=0}^{\infty} \frac{\beta^n(2P)^{-n}}{n!} \hat{U}^n \ket{x_k} \\
    &= \sum_{n=0}^{\infty} \frac{\beta^n(2P)^{-n}}{n!} (U(x_k))^n \ket{x_k} \\
    &= e^{-\beta U(x_k)/2P} \ket{x_k}
  \end{split}
\ee
So equation (\ref{eq:omega_mat}) simplifies to
\be
  \braket{x_{k+1}|\Omega|x_k} = e^{-\beta U(x_{k+1})/2P} \braket{x_{k+1}|e^{-\beta \hat{K}/P} |x_k} e^{-\beta U(x_k)/2P} .
\ee
We know that $\hat{K}$ is a function of the momentum operator, $\hat{K} = \hat{p}^2/2m$, so we can use the same technique as we did with $\hat{U}$ and the position eigenvectors by introducing the resolution of identity in terms of the momentum basis
\be
  \begin{split}
    \braket{x_{k+1}|\Omega|x_k} &= \int e^{-\beta U(x_{k+1})/2P} \bra{x_{k+1}}e^{-\beta \hat{K}/P} \ket{p}\braket{p|x_k} e^{-\beta U(x_k)/2P} \\
    &= \int e^{-\beta U(x_{k+1})/2P} e^{-\beta p^2/2mP} \braket{x_{k+1}|p}\braket{p|x_k} e^{-\beta U(x_k)/2P} dp
  \end{split}
\ee
where $\braket{x_{k+1}|p}$ is the momentum wavefunction expressed in the coordinate $x_{k+1}$, and $\braket{p|x_k}$ is the complex conjugate of the momentum wavefunction expressed in the coordinate $x_k$ basis, i.e. $\braket{p|x_k} = \braket{x_k|p}^*$.

We can obtain a functional form of these wavefunctions by recalling the eigenvalue problem for the momentum operator,
\be
  \hat{p} \ket{p} = p \ket{p}
\ee
If we project this equation on the coordinate basis, we can make use of the hermitian property of the momentum operator to obtain
\be
  \braket{x|\hat{p}|p} = p \braket{x|p} \quad \Rightarrow \quad \braket{\hat{p} x|p} =  p \braket{x|p} \quad \Rightarrow \quad -i \hbar \frac{\partial}{\partial x} \braket{x|p} = p \braket{x|p}
\ee
And if we write $\braket{x|p} = \psi_p(x)$, then we just have a first order differential equation.
The solution to this equation can easily be found
\be
  - i \hbar \frac{\partial}{\partial x} \psi_p(x) = p \psi_p(x) \quad \Rightarrow \quad \int \frac{d \psi_p}{\psi_p} dx = \int \frac{ip}{\hbar} dx \quad \Rightarrow \quad \ln \psi_p(x) = \frac{ipx}{\hbar} + C_1 \quad \Rightarrow \quad \psi_p(x) = C_2 e^{ipx/\hbar}
\ee
where $C_2$ is a normalization constant.
With proper normalization, the momentum wavefunction looks like
\be
  \braket{x|p} = \psi_p(x) = \frac{1}{\sqrt{2 \pi \hbar}} e^{ipx/\hbar} .
\ee
And it's complex conjugate looks like,
\be
  \braket{x|p}^* = \braket{p|x} = \frac{1}{\sqrt{2 \pi \hbar}} e^{-ipx/\hbar}
\ee
We can use this to rewrite our matrix elements of $\Omega$ as
\be
  \begin{split}
    \braket{x_{k+1}|\Omega|x_k} &= \int e^{-\beta U(x_{k+1})/2P} e^{-\beta p^2/2mP} \braket{x_{k+1}|p}\braket{p|x_k} e^{-\beta U(x_k)/2P} dp \\
    &= \frac{1}{2 \pi \hbar} \int e^{-\beta p^2/2mP} e^{-\beta(U(x_{k+1}) + U(x_k))/2P} e^{ip(x_{k+1} - x_k)/\hbar} dp
  \end{split}
\ee
We can move the exponential without the momenta outside of the integral and combine the exponentials with the momenta to obtain
\be
  \braket{x_{k+1}|\Omega|x_k} = \frac{1}{2 \pi \hbar} e^{-\beta(U(x_{k+1}) + U(x_k))/2P} \int e^{-\beta p^2/2mP + ip(x_{k+1} - x_k)/\hbar} dp
\ee
Since our momentum basis extends from $+ \infty$ to $- \infty$, we end up having an integral of a gaussian if we complete the square,
\be
  \begin{split}
    \frac{- \beta p^2}{2mP} + \frac{(x_{k+1} - x_k)ip}{\hbar} &= \frac{- \beta}{2mP} \Big[ p^2 - \frac{2imP(x_{k+1} - x_k)}{\beta \hbar} \Big] \\
    &= \frac{- \beta}{2mP} \Big[ \Big( p -  \frac{mP(x_{k+1} - x_k)i}{\beta \hbar} \Big)^2 - \Big( \frac{imP(x_{k+1} - x_k)}{\beta \hbar} \Big)^2 \Big] \\
    &= \frac{- \beta}{2mP} \Big[ \Big( p -  \frac{imP(x_{k+1} - x_k)}{\beta \hbar} \Big)^2 + \frac{m^2P^2(x_{k+1} - x_k)^2}{\beta^2 \hbar^2} \Big] \\
    &= \frac{- \beta}{2mP} \Big( p -  \frac{imP(x_{k+1} - x_k)}{\beta \hbar} \Big)^2 - \frac{mP(x_{k+1} - x_k)^2}{2 \beta \hbar^2} \\
  \end{split}
\ee
and make a variable substitution,
\be
  u = p -  \frac{imP(x_{k+1} - x_k)}{\beta \hbar} .
\ee
Our integral is now,
\be
  \braket{x_{k+1}|\Omega|x_k} = \frac{1}{2 \pi \hbar} \text{exp} \Big[ \frac{-\beta(U(x_{k+1}) + U(x_k))}{2P} \Big] \text{exp} \Big[ \frac{-mP(x_{k+1} - x_k)^2}{2 \beta \hbar^2} \Big] \int e^\frac{-\beta u^2}{2mP} du
\ee
which evaluates to,
\be
  \begin{split}
    \braket{x_{k+1}|\Omega|x_k} &= \frac{1}{2 \pi \hbar} \text{exp} \Big[ -\frac{\beta}{2P}(U(x_{k+1}) + U(x_k)) \Big] \text{exp} \Big[ -\frac{mP}{2 \beta \hbar^2}(x_{k+1} - x_k)^2 \Big] \Big( \frac{2 \pi m P}{\beta} \Big)^{1/2} \\
    &= \Big( \frac{mP}{2 \pi \beta \hbar^2} \Big)^{1/2} \text{exp} \Big[ \frac{mP}{2 \beta \hbar^2}(x_{k+1} - x_k)^2 - \frac{\beta}{2P}(U(x_{k+1}) + U(x_k)) \Big]
  \end{split}
\ee
Plugging in our matrix elements into equation (\ref{eq:density_omega}) P times for each element gives us the discretized path integral representation of our density matrix.
\be
  \rho(x,x') = \lim_{P \to\infty} \Big( \frac{mP}{2 \pi \beta \hbar^2} \Big)^{P/2} \int \cdots \int dx_P \cdots dx_2 \quad \text{exp}\Big( \sum_{k=1}^P \Big[ \frac{mP}{2 \beta \hbar^2}(x_{k+1} - x_k)^2 - \frac{\beta}{2P}(U(x_{k+1}) + U(x_k)) \Big] \Big)
\ee
where $x_1 = x$ and $x_P = x'$.
\end{document}
